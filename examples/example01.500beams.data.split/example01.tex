%   This is the only one tex file of example01.data.split for
%       https://github.com/LiuGangKingston/FORTRAN-CSV-TIKZ.git
%            Version 2.0
%   free for non-commercial use.
%   Please send us emails for any problems/suggestions/comments.
%   Please be advised that none of us accept any responsibility
%   for any consequences arising out of the usage of this
%   software, especially for damage.
%   For usage, please refer to the README file.
%   This code was written by
%        Gang Liu (gl.cell@outlook)
%                 (http://orcid.org/0000-0003-1575-9290)
%          and
%        Shiwei Huang (huang937@gmail.com)
%   Copyright (c) 2021
%
%   This example is for drawing trajectories of laser beams striking 
%   on point A of a spherical segment (made of glass with index of 
%   refraction n=1.5) by using Tikz. The radius of the whole sphere 
%   is r=8 cm, the distance from the striking point A to point B is 
%   a=3 cm. The angle ABD is 90 degrees. The distance from center C 
%   to the bottom side of the spherical segment is b=3 cm. The angle 
%   CFD is also 90 degrees. For incident angles being 3, 13, 23, 33, 
%   43, and 53 degrees respectively, for examples, the refracted, 
%   then reflected, and further refracted beams are also drawn. All 
%   computations are done with the FORTRAN language.
%
%
%
%
%
%   Since data split into smaller files, additional 
%   command "\foreach \myfile in {
%   iterated.alldata.1.csv, iterated.alldata.2.csv, 
%   iterated.alldata.3.csv, iterated.alldata.4.csv, 
%   iterated.alldata.5.csv, iterated.alldata.6.csv, 
%   iterated.alldata.7.csv, iterated.alldata.8.csv, 
%   iterated.alldata.9.csv,iterated.alldata.10.csv} { ...} " is added. 
%   Hope less computer memory is needed.
%
%
%
% 
%
\documentclass[tikz,border=5mm]{standalone}
%\documentclass{article}
\usepackage{tikz,csvsimple}
\usetikzlibrary{calc}


\begin{document}

\begin{tikzpicture}

% Initial data and setup

\coordinate (circlecenter) at (0,0);

% To setup/initialize 
\csvreader[head to column names]{setup.scalars.csv}{}
{
% A few more parameters for graph drawing
\pgfmathsetmacro{\ra}{\bigradius*0.7}
\pgfmathsetmacro{\rs}{\bigradius+\ra}
\pgfmathsetmacro{\rj}{\bigradius*0.1}
\pgfmathsetmacro{\rk}{\bigradius*0.15}
\pgfmathsetmacro{\rjj}{\bigradius*(0.15)}
\pgfmathsetmacro{\rkk}{\rjj+0.18}

% The half circle:
\draw [dashed, line width=0mm] ($(0:\bigradius)+(circlecenter)$) arc [start angle=0, end angle=180, radius=\bigradius] -- cycle;
% The circle center:
\node at ($(circlecenter) + (-0.4,0)$) [above left]  {\textbf{C}} ;

% The glass matetial:
\coordinate (pz) at (\z,\b);
\draw [very thick, fill=blue!20] ($(pz)+(circlecenter)$) arc [start angle=\anglez, end angle=180-\anglez, radius=\bigradius] -- cycle;
\node at (-2,6.2) {Refractive index = \refractiveindex} ;

% The incident point A
\coordinate (positiona) at (\c, \a+\b);
\draw [dashed] (circlecenter) -- + (\anglea:\rs);
\node at ($ (positiona) + (0,0.5) $) {\textbf{A}};

% Lines a and b
\draw [dashed] (positiona) -- node [right] {a} + (0,-\b) node [below] {\textbf{B}};
\draw [dashed] (circlecenter) -- node [right] {b} + (0,\b) node [below left] {\textbf{F}};



% To find E position by solving equations, with t as DE length: 
%  (t sin + \b)^2 + (t cos + \dx )^2 = 64
%  t^2 + 2 (\b sin + \dx cos ) + (\b sin + \dx cos )^2 
%  = 64 - \b^2 - \dx^2 + (\b sin +  \dx cos  )^2 
%  t = sqrt ((\b sin + \dx cos )^2 +  64 - \b^2 - \dx^2 )  - (\b sin + \dx cos ) 

\foreach \myfile in {iterated.alldata.1.csv, iterated.alldata.2.csv, iterated.alldata.3.csv, iterated.alldata.4.csv, 
iterated.alldata.5.csv, iterated.alldata.6.csv, iterated.alldata.7.csv, iterated.alldata.8.csv, iterated.alldata.9.csv, iterated.alldata.10.csv}
{
\csvreader[head to column names]{\myfile}{}
{
% Point A area
\draw [\mycolor,line width=0mm, <-] (positiona) -- + (\anglea+\incidentangle:\ra); 


% Point D area
\coordinate (positiond) at (\dx,\b);
\draw [\mycolor, line width=0mm] (positiona) -- (positiond); 

% Point E area
\coordinate (positione) at (\ex, \ey);
\draw [\mycolor, line width=0mm, ->] (positiond) -- (positione); 
\draw [\mycolor, line width=0mm, ->] (positione) -- + (\anglece-\outangle:\ra); 



\pgfmathparse{(\totallines < 11) ? int(1) : int(0)}
\ifnum\pgfmathresult=1 
        \pgfmathsetmacro{\rjj}{\bigradius*(0.05+0.06*\i)}
        \pgfmathsetmacro{\rkk}{\rjj+0.18}

        \draw [line width=0mm] ($ (positiona) + (\anglea:\rjj) $) arc  [start angle=\anglea, delta angle=\incidentangle, radius=\rjj] ;
        \node at ($ (positiona) + (\incidentangle*0.5+\anglea:\rkk) $) {$\theta_1$};

        \draw [dashed] (positiond) -- + (0,.4);
        \node at  (positiond) [below]  {\textbf{D}};

        \draw [dashed] (circlecenter) -- + (\anglece:\rs);
        \draw [line width=0mm] ($ (positione) + (\anglece:\rj) $) arc  [start angle=\anglece, delta angle=-\outangle, radius=\rj] ;
        \node at ($ (positione) + (-\outangle*0.5+\anglece:\rk) $) {$\theta_4$};
        \node at (positione) [above] {\textbf{E}};
        \node at (0,-1 -\i*0.6 ) {Beam \  \i \ of \totallines : incident angle $\theta_1=\incidentangle$, out angle  $\theta_4=\outangle$};
\fi


}
}
}

\end{tikzpicture}


\end{document}



